% SECTION: TÁBLÁZATOK    
\section{Táblázatok használata}\label{sec:tablazatok}    

	\begin{table}[!h]
		\begin{center}
        	\begin{tabular}{|l|ccc|}
            \hline
        	Számrendszer  &  Alap &	Jele      & Példa \\
            \hline
			Decimális 	  &  10   &	          & 139  \\
			Bináris 	  &  2 	  & b 	      & 100b \\
			Oktális 	  &  8    & 0 	      & 065 \\
			Hexadecimális &  16   &	0x vagy h & 0x243, 22h\\
            $\pi$-alapú   &  $\pi$ & $\cdots$ & $\cdots$ \\
            \hline
        	\end{tabular}		
		\end{center}
        \label{tab:szamrendszerek}
        \caption[Első táblázat]{A félév elején tanult számrendszerek\dots}
	\end{table}
    
    A babel csomag használatával névelővel is elláthatjuk
    a referenciákat.
    Az utolsó sornál nem kell sortörést jelezni, 
    csak ha vonalat húzunk utána\dots.
    
    Megjegyzés: a táblázat annál szebb, minél kevesebb a vonal
    és ha függőleges nincs, vagy alig észrevehető.
    
    Megjegyzés2: további csomagok segítségével tudunk
    szaggatott vonalat és egyéb vastagságú vonalakat használni.
    
    \subsection{Oszlopok összehúzása}
    	\begin{table}[!h]
		\begin{center}
        	\begin{tabular}{|lccc|}
            \hline
            \multicolumn{4}{|c|}{Közös oszlop}\\
            \hline
        	Számrendszer  &  Alap &	Jele      & Példa \\
            \hline
			Decimális 	  &  10   &	          & 139  \\
			Bináris 	  &  2 	  & b 	      & 100b \\
			Oktális 	  &  8    & 0 	      & 065 \\
			Hexadecimális &  16   &	0x vagy h & 0x243, 22h\\
            $\pi$-alapú   &  $\pi$ & $\cdots$ & $\cdots$ \\
            \hline
        	\end{tabular}		
		\end{center}
	\end{table}
    
\section{Elágazások}

 \begin{equation}	
 f(n) =
  \begin{cases}
    n/2       & \quad \text{ha } n \text{ páros}\\
    -(n+1)/2  & \quad \text{ha } n \text{ páratlan}\\
  \end{cases}
  \end{equation}
  
% SECTION: EGYÉB  
\section{Egyéb tudnivalók} 

	Órai munka: keressetek rá a neten, hogy lehet algoritmust
    közölni latex-ben.
	\footnote{Lábjegyzetet írunk}
    

\section{Idézés}

	Minden tudományos munkában a felhasznált irodalmat idézzünk.
    Soha nem használunk fel irodalmat anélkül, hogy idéznénk.
    Soha.
    Itt egy idézet \cite{nika2016strong}. Ennyi.
    
    Megtudtam, hogy lehet a magyar babel csomagot használni:
    recompile from scratch.
    
    
\bibliographystyle{plain}
\bibliography{bibliography.bib}